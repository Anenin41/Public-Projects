% Conclusion of the Final Essay for Math & Environment %
% Author: Konstantinos Garas
% E-mail: kgaras041@gmail.com // k.gkaras@student.rug.nl
% Created: Fri 28 Mar 2025 @ 16:23:50 +0100
% Modified: Sat 29 Mar 2025 @ 13:48:21 +0100

\section{Conclusion}
\label{sec: conclusion}

To conclude, this case study has introduced the famous algorithmic families that are used to solve the Travelling Salesman Problem. Each heuristic family is comparable and stems from various different concepts already in place in nature. Unfortunately for us, this doesn't mean that the great ideas described in this report can have any effect at solving other similarly hard problem of the scientific world.

As a final remark, I would like to close the case study by my personal opinion on the subject. Overall, as a scientist, I like both the Simulated Annealing and the Ant Colony algorithms. The first, because I find it very easy to combine it with other well established algorithms in the field of discrete optimization \cite{ye2013improved}, while the latter because of its ingenious observation of how ants search for food.

Additionally, a more efficient manual implementation of these algorithms, for example in \texttt{C} or \texttt{Rust}, or by someone with a more refined coding experience, would produce results much more compatible with standard academic practices. Unfortunately, the timetables of such a project is out of the scope of 1 week, which was the time used to develop the source files.
