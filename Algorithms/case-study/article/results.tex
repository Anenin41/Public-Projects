% Results of the Final Essay for Math & Environment %
% Author: Konstantinos Garas
% E-mail: kgaras041@gmail.com // k.gkaras@student.rug.nl
% Created: Fri 28 Mar 2025 @ 16:05:36 +0100
% Modified: Sat 29 Mar 2025 @ 13:48:13 +0100

\section{Comparison of the Algorithms}
\label{sec: results}
The heuristic algorithms that were described in Section \ref{sec: heuristics} are now compared in terms of their performance for 30 randomly generated graphs of 229 cities, a size of which is equivalent to the size of the Netherlands. 

Table \ref{table: nicknames} provided a legend for the nicknames of the algorithms in the figures, while figures \ref{fig: mean_distance}, \ref{fig: mean_uptime}, \ref{fig: min_distance} compare the algorithms in terms of \textit{mean distance, mean uptime, minimum distance achieved} respectively. Lastly, figure \ref{fig: compare_all} groups all the results together for a more cohesive view.

\subsection{Technical Information}
The algorithms were implemented in \texttt{Python 3.12.3} with the only materials to be used being the pseudo-codes in the original papers. Then, the case study was run in the Hábrók High Performance Computing cluster of the University of Groningen, over the period of 16 hours, with 5 processing cores working at 100\%. Due to the intense nature of the computations, I strongly recommend not to run these programs in a normal computer.

\begin{table}[htbp]
	\centering
	\begin{tabular}{|c|c|}
		\hline
		\textbf{Algorithm}	&	\textbf{Figure Name} \\
		\hline
		Nearest Neighbour  & NN \\
		Baseline Genetic Algorithm	&	BGA	\\
		Genetic Algorithm	&	GA	\\
		Greedy Permuting Method	&	GPM	\\
		Baseline Simulated Annealing	&	BSA	\\
		Simulated Annealing	&	SA	\\
		Baseline Ant Colony Optimization	&	BACO	\\
		Ant Colony Optimization	&	ACO \\
		\hline
	\end{tabular}
	\caption{Table of names used in the following figures.}
	\label{table: nicknames}
\end{table}

\begin{itemize}
	\item \textbf{Baseline GA, SA, ACO:} The results are generated without fine-tuning of the user-defined parameters\footnote{The Nearest Neighbour algorithm doesn't accept user defined parameters, and as such, it can't be fine-tuned}.
	\item \textbf{GA, SA, ACO:} These algorithms have been fine-tuned in terms of the optimal output. Speed was entirely sacrificed in order to get the best possible final tour in each test.
\end{itemize}

\begin{figure}[htbp]
	\centering
	\includegraphics[width=0.7\textwidth]{Extras/Comparison_Mean_Distance.png}
	\caption{Comparison of the heuristic algorithms in terms of Mean Distance, after 30 randomly generated experiments. Fine-tuned Ant Colony Optimization produced the best results.}
	\label{fig: mean_distance}
\end{figure}

\begin{figure}[htbp]
	\centering
	\includegraphics[width=0.7\textwidth]{Extras/Comparison_Mean_Uptime.png}
	\caption{Comparison of the Mean Uptime of the heuristic algorithms, after 30 randomly generated experiments. Here, NN is the fastest of all other choices by a large margin.}
	\label{fig: mean_uptime}
\end{figure}

\begin{figure}[htbp]
	\centering
	\includegraphics[width=0.7\textwidth]{Extras/Comparison_Min_Distance.png}
	\caption{Comparison of the Minimum Distance achieved by the heuristic algorithms, after 30 randomly generated experiments. Once more, Ant Colony Optimization produced the best possible answer.}
	\label{fig: min_distance}
\end{figure}

\begin{figure}[h]
	\centering
	\includegraphics[width=0.7\textwidth]{Extras/Comparison_All.png}
	\caption{A final figure showcasing the Mean Distance, and Mean Uptime achieved for each heuristic algorithm in the case study.}
	\label{fig: compare_all}
\end{figure}
