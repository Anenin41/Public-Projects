% File containing the Bonus Exercise of CFD - Practical 4 %
% Author: Konstantine Garas
% E-mail: kgaras041@gmail.com // k.gkaras@student.rug.nl
% Created: Wed 20 Nov 2024 @ 11:27:29 +0100
% Modified: Wed 20 Nov 2024 @ 13:38:48 +0100

\subsection{(Bonus).}
In this exercise, I am tasked with deriving the stability requirements for the general Crank-Nicolson method of the unsteady convection diffusion equation, using the central discretization for the spatial derivatives. I start by noting down the unsteady convection diffusion equation.

\[
	\frac{d \phi}{d t} + U \frac{d \phi}{d x} = k \frac{d^2 \phi}{dx^2}
\]

I will reform this equation into a discrete numerical scheme by using the generalized Crank-Nicolson method for the time derivative, and the central discretization for the spatial derivative. A major observation to be made is that the Crank-Nicolson method weights between explicit and implicit time integration methods. Without further ado, the following equations generate the discrete scheme.

\begin{align*}
	\frac{d \phi}{d t} &= \frac{\phi_{i}^{(n+1)} - \phi_{i}^{(n)}}{\delta t} \\
	\frac{d \phi}{d t} &= \left[ (1-\omega) \frac{\phi_{i+1}^{(n)} - \phi_{i-1}^{(n)}}{2h} + \omega \frac{\phi_{i+1}^{(n+1)} - \phi_{i-1}^{(n+1)}}{2h} \right] \\
	\frac{d^2 \phi}{dt^2} &= \left[ (1-\omega) \frac{\phi_{i+1}^{(n)} - 2 \phi_{i}^{(n)} + \phi_{i-1}^{(n)}}{h^2} + \omega \frac{\phi_{i+1}^{(n+1)} - 2 \phi_{i}^{(n+1)} + \phi_{i-1}^{(n+1)}}{h^2} \right]
\end{align*}

By plugging in these equations to the differential equation at the start of this discussion, and by grouping over implicit and explicit terms, I end up with the following relation.

\begin{gather*} 
	\omega \left( \frac{\eta}{2} - \frac{d}{2} \right) \phi_{i+1}^{(n+1)} + (1 + \omega d) \phi_{i}^{(n+1)} + \omega \left( -\frac{\eta}{2} -\frac{d}{2} \right) \phi_{i-1}^{(n+1)} = \\
	= ( 1 - \omega ) \left( \frac{d}{2} - \frac{\eta}{2} \right) \phi_{i+1}^{(n)} + \left[ 1 - (1 - \omega)d \right] \phi_{i}^{(n)} + (1-\omega) \left( \frac{d}{2} + \frac{\eta}{2} \right) \phi_{i-1}^{(n)} \numberthis \label{eq: bonus}
\end{gather*}

By using the same argument as in \ref{subsec: 4f}, I can rewrite equation \ref{eq: bonus} in matrix form.

\[
	A \cdot \vec{\Phi}^{(n+1)} = B \cdot \vec{\Phi}^{(n)}
\]
and for the method to be numerically stable, I ask the coefficients of matrices \(A, B\) to not be negative. This observation gives rise to the following set of conditions, all of which must be satisfied in order to not have instabilities in the profile of the numerical solution.

\noindent
\begin{minipage}{0.45\textwidth} % First column
	\begin{align*}
		\omega \left( \frac{\eta}{2} - \frac{d}{2} \right) &\leq 0 \\
		1 + \omega d &\geq 0 \\
		\omega \left( -\frac{\eta}{2} - \frac{d}{2} \right) &\leq 0
	\end{align*}
\end{minipage}
\hfill \& \hfill
\begin{minipage}{0.45\textwidth} % Second column
	\begin{align*}
		(1 - \omega) \left( \frac{d}{2} - \frac{\eta}{2} \right) &\geq 0 \\
		1 - (1-\omega)d &\geq 0 \\
		(1  - \omega) \left(\frac{d}{2} + \frac{\eta}{2} \right) &\geq 0
	\end{align*}
\end{minipage}

The \(R\) and \(L\) coefficients of the implicit matrix are negative by construction. However, because numerically the matrix must be inverted to get a solution, they contribute positively to the result. Moreover, by analyzing the implicit conditions, I get:

\[
	\eta \leq d
\]
and
\[
	\omega \geq 0
\]
because \( \omega \) represents the bias of explicit or implicit method, it makes no sense numerically to be negative. More useful results are generated by the explicit conditions. Those are the following:

\begin{gather*}
	\omega \leq 1 \\
	\eta \leq d \\
	1 -\omega \leq \frac{1}{d}
\end{gather*}

By combining everything, the numerical stability conditions that I was searching for are:

\begin{gather*}
	0 \leq \omega \leq 1 \\
	0 \leq \eta \leq d \\
	1 - \omega \leq \frac{1}{d}
\end{gather*}
and, if these hold, the method will not have numerical instabilities, and as a result it will be wiggle-free. Lastly, I also need to check what is the case for damping. For this, I need the Fourier symbol.

\[
	\phi_{i}^{(n)} = \hat{\phi}^n e^{i \theta j } \,\, \text{ with } \,\, \theta = \beta h
\]

I will plug the Fourier symbol in \ref{eq: bonus}, and after deleting common terms in both sides, I end up with the following equation.

\begin{gather*}
	\hat{\phi}^{n+1} \left[ \omega \frac{\eta}{2} \left( e^{i\theta} - e^{-i\theta} \right) - \omega \frac{d}{2} \left( e^{i\theta} + e^{-i\theta} \right) + \omega d + 1 \right] = \\
	= \hat{\phi}^{n} \left[ (1-\omega) \frac{d}{2} \left( e^{i\theta} + e^{-i\theta} \right) - (1 - \omega) \frac{\eta}{2} \left( e^{i\theta} - e^{-i\theta} \right) + 1 - (1-\omega)d \right]
\end{gather*}

To simplify this further, I am going to use the following trigonometric identities.

\begin{align*}
	e^{i\theta} + e^{-i\theta} &= 2 \cos\theta \\
	e^{i\theta} - e^{-i\theta} &= 2i \sin\theta
\end{align*}

By plugging in these equations and dividing everything by \( \hat{\phi}^{n} \), I end up with
\[
	\frac{\hat{\phi}^{n+1}}{\hat{\phi}^{n}} = \frac{1 - (1-\omega)d(1 - \cos\theta ) - i (1-\omega)\eta\sin\theta}{1 + \omega d (1 - \cos\theta) + i \omega \eta \sin\theta}
\]
this is a complex fraction, and without a specific value for \( \omega \), to study it using the method that multiplies everything by the complex conjugate of the denominator, will be really difficult, because then the modulus will have many terms and be simultaneously under a square root. However, to get a result about Fourier stability, it is also imperative to study if

\[
	|g(\theta)| \leq 1
\]
holds.

So to get a meaningful and not cumbersome result, I have to act smart, and notice the special structure of the complex fraction. In order to not have numerical instability, I have already mentioned that \( \eta \leq d \). What happens when the diffusion term dominates in the physics of the problem, i.e., what happens when \( \eta << d \)? Then, the complex fraction drops the imaginary parts of the numerator and the denominator, and it is transformed to 

\[
	g(\theta) \approx \frac{1 - (1 - \omega)d(1-cos\theta)}{1 + \omega d (1-cos\theta)}
\]

For Fourier stability to be satisfied, I need

\[
	1 - (1 - \omega)d (1-\cos\theta) \geq 0
\]

By introducing one more trigonometric identity
\[
	1 - \cos\theta = 2 \sin^2{\left( \frac{\theta}{2} \right)}
\]
then, the inequality above becomes

\[
	1 - 2 (1 - \omega)d \sin^2{\left( \frac{\theta}{2} \right)} \geq 0
\]
and the worst case scenario happens when \(\theta = \pi\). In that case,
\[
	1 - 2(1 - \omega)d \geq 0 \implies d \leq \frac{1}{2(1-\omega)}
\]
and this is the relation that I have to satisfy in order to have Fourier stability in the solution, i.e., have a bounded Fourier amplification factor.
